\documentclass[]{report}
\usepackage{harvard}

\begin{document}

\begin{center}
    \noindent \Huge{\bfseries Mohamed El Shorbagy}\\[.4em]
    \large
        \textbf{Email:} mohrizq895@gmail.com 
    |   \textbf{Tel:} $+20 \ 1222448102$
    |   \link{GitHub}{\github}
    |   \link{LinkedIn}{\linkedin}
    |   \link{Website}{\site}   
\end{center}

\vspace{2mm}

\large
\Title{Education}

\entry
    {2020 - 2025} 
    {Bachelor of Science in Computer \& Systems Engineering }
    {Ain Shams University}
    {Cairo, Egypt}
    {\tb Senior year 2 (year 5/5) | \textbf{GPA}: 3.62 / 4.00 | \textbf{Class rank: 4/160} \\  
     \tb \textbf{Relevant Courses:} Compiler Design, Operating Systems, Embedded Systems, Network Security, Discrete Math, Image Processing, Machine Learning, Deep Learning, Database Systems, Artificial Intelligence, Software Engineering, Distributed Systems, Data Structures, Design of Algorithms.\\
     \tb \textbf{Thesis:} \emph{\textbf{Dental Crown Generation with Generative AI}}, sponsored by \linkk{Atomica.ai}{https://atomica.ai/}\\
     Contributions:\\
     \hspace*{5mm} \tb Developed a geometric kernel in Rust with Python bindings for accelerated mesh operations.\\
     \hspace*{5mm} \tb Implemented a dental arch (Occlusion) alignment algorithm.
    }  

\vspace{2mm}

\Title{Technical Skills}

\begin{tabular}{ l l l l l}
      \bf{- Programming-Languages:} & Python, & Lua, & C++, & Rust\\ 
      \bf{- Graphics:} & OpenGL,&  SDL, & ASSIMP \\ 
      % \bf{- Scientific Computing:} & Octave, & NumPy, & SciPy, & Matplotlib \\ 
      \bf{- Tools \& Platforms:} & Linux, & Bash, & CMake, & git \\ 
      % \bf{- Databases \& Data stores :} & MySQL, & TinyDB, & Pandas \\ 
       
\end{tabular}

\vspace{4mm}

\Title{Experience}

\entry
    {Oct 2023 - Feb 2024}
    {Software Engineering Intern}
    {ASMARINE (Autonomous Underwater Vehicles team)}
    {ASU, Cairo, Egypt}
    {
      \tb Implemented state-of-the-art algorithms in SLAM \& Computational Geometry. \\
      \tb Optimized code for resource-constrained computers.
    }

\entry
    {Jun - Sep 2023}
    {Undergraduate Research Assistant}
    {Human Centered Mechatronics Lab}
    {ASU Virtual Hospitals, ASU}
    {
      \tb Implemented a TCP communication tunnel to retrieve data from various sensors through XML. \\
      \tb Synchronized motion capture cameras with the metabolic energy consumption system. \\
      \tb Automated sensor calibration process.
    }

\entry
    {Aug - Oct 2022}
    {Optimization \& Signal Processing Intern}
    {Dynamic Systems \& Digitalization cluster - Cardiff University}
    {ASU, Cairo}
    {
      \tb Utilized the Akaike Information Criterion estimator for precise determination of signal onset time.\\
      \tb Implemented TDOA algorithm with particle swarm optimization to localize acoustic sources.
    }


\Title{Personal Projects}

\entry
    {Sep 2024}
    {mesher: 3D Triangular mesh viewer and inspector, \link{Code}{\github/mesher}}
    {}{}
    {
        \tb Implemented a mesh viewer with OpenGL API, supporting lighting, and camera movement.\\
        \tb Implemented triangle selection algorithm with BVH and Möller–Trumbore algorithm.
    }

\entry
    {Aug 2024}
    {mark: CLI-based Bookmark manager, \link{Code}{\github/mark}}
    {} {}
    {
        \tb Developed a CLI tool for global bookmark management built on top of rofi.\\ 
        \tb Implemented client-server architecture with async sockets for Rofi communication.\\
        \tb Created a wrapper around TinyDB with orjson for efficient bookmark storage and querying.\\ 
        \tb Added a parser for the Netscape bookmark file format and various export options.
    }

\entry
    {Nov 2023}
    {automata-cli: Automata Renderer and Minimizer, \link{Code}{\github/automata-cli}}
    {} {}
    {
        \tb Built a CLI tool to parse and manipulate program-like automata specifications. \\ 
        \tb Enabled minimization, format conversion, and custom algorithm manipulation. \\ 
        \tb Supported rendering automata into various formats for document embedding.
    }

\entry
    {Feb 2023}
    {cv.py: YAML to TeX Adapter, \link{Code}{\github/CV.py}}
    {} {}  
    {
        \tb Created a CLI tool to easily convert YAML files into LaTeX-based CVs.\\
        \tb Enabled users to focus on content creation while the tool manages the formatting process.\\
        \tb Supported CV compilation through either a cloud-based LaTeX compiler or local compilation.
    }

\entry
    {Feb 2023}
    {Implementation of A* on Open Street Maps Data, \link{Code}{\github/Astar-OSM}}
    {} {}    
    {
       \tb Implemented A* algorithm with KDTree and KNN to compute the shortest path on OSM data.\\
       \tb Utilized the randomized median of medians algorithm to accelerate KDTree construction. \\
       \tb Developed the algorithm on top of an extended Kalman filter and embedded microcomputer.
    }

\Title{Open-Source Contributions}

\begin{tabular}{ll}
- \linkk{NetworkX}{https://github.com/networkx/networkx/pulls?q=is\%3Apr+author\%3Amohamedrezk122+}
 & a network analysis library and graph theoretic algorithms in Python. \\
- \linkk{SymPy}{https://github.com/sympy/sympy/pulls?q=is\%3Apr+author\%3Amohamedrezk122+}
 & a computer algebra \& symbolic computation in Python.
\end{tabular}

\vspace{4mm}


\Title{Hackathons \& Competitions}

\entry
    {Summer 2023}
    {NASA Space Apps Cairo}
    {The American University in Cairo }{}
    {
        \tb Developed a project centered on data sonification, enhancing the perception of space imagery.\\
        \tb Implemented a melody fitting algorithm: aligning classical music pieces with the input image.\\
        \tb Received the \textbf{"Most Innovative Solution"} award and \textbf{25k EGP} prize.
    }
    
\entry 
    {Summer 2022}
    {NASA Space Apps Cairo}
    {The American University in Cairo }{}
    { 
        \tb Developed a web interface for ISS 3D virtual tracking in real-time. \\
        \tb Implemented orbital propagation algorithm and a sun tracking algorithm for ISS solar panels.\\
        \tb Awarded 500\$ AWS Credit Points Prize.
    }

\Title{Awards \& Honors}

\entry
    {July 2024}
    {SciPy 2024 Conference}  
    {SciPy, NumFocus}
    {Tacoma, WA, USA}
    {
        \tb Selected to attend SciPy Conference 2024 with full financial aid.\\
        % \tb Recognized for impact and dedication in advancing open source projects.\\
        \tb Engaged with leading experts in the field of scientific and high performance computing.
    }

\entry
    {Summer 2022}
    {Top 100 entries \& Top 25 Articles}
    {Summer of Math Exposition (SoME\#2)}
    {3Blue1Brown \& Leios Labs}  
    {
        % \tb Participated in a competition aimed at presenting in-depth content in mathematics, computer science, and physics through engaging mediums.\\
        \tb Secured a spot among the top 100 overall submissions.\\
        \tb Ranked in the top 25 for non-video submissions (e.g., articles and games).
    }

\Title{Articles}

\entry
    {Jul 2023}
    {The Generalization of Fifteen Puzzle, 
    \link{Article}{\site/fifteen-puzzle}\link{Code}{\github/fifteen-puzzle-solver}}  
    {} {}
    {
      \tb Discussed the generalization and solvability of the fifteen puzzle with group theory.\\
      \tb Implemented a PQ solver using weighted iterative deepening A*.
    } 

\entry
    {May 2022}
    {Diffusion Equation: A computational approach, \link{Article}{\site/diffusion}}  
    {} 
    {}
    {
        \tb Analyzed the diffusion equation as both a mathematical and computational model.\\
        \tb Worked on generalizing results to higher dimensions.    
    }

\end{document}
