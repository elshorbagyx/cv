\documentclass[hidelinks]{report}
\usepackage{harvard}

\begin{document}

\begin{center}
    \noindent \Huge{\fontfamily{lmss} \selectfont \bfseries Mohamed El Shorbagy}\\[.4em]
    \large
    % 2002114@eng.asu.edu.eg \\[.2\baselineskip]
    $+20 \ 1222448102$           |
    mohrizq895@gmail.com         | 
    \underline{\url{\github}}    |
    \underline{\url{\linkedin}}  |
    \underline{\url{\blog}}  
\end{center}


\vspace{2mm}

\large
\Title{Education}

\entry
    {2020 - present} 
    {Bachelor of Science in Computer Engineering }
    {Ain Shams University}
    {Cairo, Egypt  }
    {\textbullet Senior year 2 (year 5/5) | Anticipated Graduation: Jul 2025 \\ 
     \textbullet \textbf{CGPA}: 3.62 / 4.0. \\  
     \textbullet \textbf{Relevant Courses:} Compiler Design, Discrete Math, Image Processing, Deep Learning,
     Operating Systems, Database Systems, Artificial Intelligence, Software Engineering, Distributed Systems, 
     Algorithms. 
    }  

% \entry
%     {Class 2020}
%     {STEM High School Certificate}
%     {Dakahlia STEM School}
%     {Gamasa, Egypt}
%     {
%     \textbullet \textbf{CGPA}: 4.0 / 4.0 (1$^{st}$ Rank)
%     }

\vspace{2mm}


\Title{Technical Skills}

\begin{tabular}{ l l l l l}
      \bf{- Programming-Languages (high proficiency):} & Python, & Lua, & C++ \\ 
      \bf{- Programming Languages (some proficiency):} & C, & Java, & Rust \\ 
      \bf{- Graphics:} & OpenGL \\ 
      \bf{- Scientific Computing:} & Octave, & NumPy, & SciPy, & Matplotlib \\ 
      \bf{- System and Scripting:} & Linux, & Bash \\ 
      \bf{- Databases \& Data stores :} & MySQL, & TinyDB, & Pandas \\ 
       
\end{tabular}

\vspace{4mm}

\Title{Experience}

\entry
    {Oct 2023 - Feb 2024}
    {AI Software Intern}
    {ASMARINE (Autonomous Underwater Vehicles team)}
    {ASU, Cairo, Egypt}
    {
      \textbullet Implemented state-of-the-art algorithms in Computer Vision and SLAM. \\
      \textbullet Explored the feasibility of machine learning-based control in AUVs. \\
      \textbullet Optimized code for resource-constrained computers.
    }

\entry
    {Jun - Sep 2023}
    {Undergraduate Research Assistant}
    {Human Centered Mechatronics Lab}
    {ASU Virtual Hospitals, ASU}
    {
      \textbullet Implemented a TCP communication tunnel to retrieve data from various sensors through XML commands. \\
      \textbullet Synchronized motion capture cameras with the metabolic energy consumption system. \\
      \textbullet Automated sensor calibration process.
    }

\entry
    {Aug - Oct 2022}
    {Optimization \& Signal Processing Intern}
    {Dynamic Systems \& Digitalization cluster - Cardiff University}
    {ASU, Cairo}
    {
      \textbullet Implemented an acoustic wave velocity algorithm within composites.\\
      \textbullet Analyzed raw sensor data to localize acoustic sources such as cracks. \\
      \textbullet Implemented optimization algorithms (particle swarm, simulated annealing, simplex method).\\
      \textbullet Explore the project: 
      \href{\github/AE-software.git}{\underline{Code}} / %
      \href{\github/AE-software/blob/master/Final-Report/Report.pdf}{\underline{Report}}
    }

\vspace{2mm}

% \entry
%     {Mar 2023}
%     {Visiting Student}
%     {Cardiff University, School of Engineering}
%     {Wales, UK}
%     {
%       \textbullet Fully funded program (UK-HE Climate Research Grant). \\
%       \textbullet Conducted Fourier analysis and signal processing on composite structures. \\ 
%     }


\vspace{2mm}
\Title{Articles}

\entry
    {Jul 2023}
    {The Generalization of Fifteen Puzzle as PQ Puzzle}  
    {} {}
    {
      \textbullet Discussed the generalization and solvability of the fifteen puzzle.\\
      \textbullet Implemented a PQ solver using weighted iterative deepening A*.\\
      \textbullet Explore the article:  \href{\blog/fifteen-puzzle}{\underline{Article}} | 
      \href{\github/fifteen-puzzle-solver}{\underline{Solver Code}}
    } 

\newpage

\entry
    {May 2022}
    {Diffusion Equation: A computational approach}  
    {} 
    {}
    {
        \textbullet Analyzed the diffusion equation as both a mathematical and computational model.\\
        \textbullet Worked on generalizing results to higher dimensions.\\
        \textbullet Explore the article: \href{\blog/diffusion}{\underline{Article}}
    }

\vspace{2mm}

\Title{Personal Projects}

\entry
    {Aug 2024}
    {mark}
    {} {}
    {
    \textbullet Developed a CLI tool for global bookmark management built on top of rofi.\\ 
    \textbullet Implemented client-server architecture with async sockets for Rofi communication.\\
    \textbullet Created a wrapper around TinyDB with orjson for efficient bookmark storage and querying.\\ 
    \textbullet Added a parser for the Netscape bookmark file format and various export options.\\
    \textbullet Navigate: \href{\github/mark}{\underline{Code}}
    }

\entry
    {Nov 2023}
    {automata-cli}
    {} {}
    {
    \textbullet Built a CLI tool to parse and manipulate program-like automata specifications. \\ 
    \textbullet Enabled minimization, format conversion, and custom algorithm manipulation. \\ 
    \textbullet Supported rendering automata into various formats for document embedding.\\
    \textbullet Explore the project: \href{\github/automata-cli}{\underline{Code}}
    }

\entry
    {Feb 2023}
    {cv.py}
    {} {}  
    {
    \textbullet Created a CLI tool to easily convert YAML files into LaTeX-based CVs.\\
    \textbullet Enabled users to focus on content creation while the tool manages the formatting process.\\
    \textbullet Supported CV compilation through either a cloud-based LaTeX compiler or local compilation.\\
    \textbullet Explore the project: \href{\github/CV.py}{\underline{Code}}
    }

\entry
    {Feb 2023}
    {Implementation of A* on Open Street Maps Data }
    {} {}    
    {
       \textbullet Implemented the A* algorithm with KDTree and KNN to compute the shortest path on OpenStreetMap data.\\
       \textbullet Utilized the Haversine heuristic and a randomized median of medians to accelerate KDTree formation. \\
       \textbullet Developed the algorithm on top of an extended Kalman filter, running on a bare-metal embedded microcontroller.\\
       \textbullet Explore the project: \href{\github/Astar-OSM}{\underline{Code}}
    }

\vspace{4mm}

\Title{Hackathons \& Competitions}

\entry
    {Summer 2023}
    {NASA Space Apps Cairo}
    {The American University in Cairo }    
    {}
    {
      \textbullet Developed a project centered on data sonification, enhancing the perception of space imagery.\\
      \textbullet Implemented a melody fitting algorithm: aligning classical music pieces with the input image.\\
      \textbullet Received the "Most Innovative Solution" award at the NASA Space Apps competition.\\
      \textbullet Awarded a prize of 25,000 Egyptian pounds.\\
    }
    
\entry 
    {Summer 2022}
    {NASA Space Apps Cairo}
    {The American University in Cairo }
    {}
    { \textbullet Developed a web interface for ISS 3D virtual tracking in real-time. \\
      \textbullet Implemented orbital propagation algorithm for International Space Station.\\ 
      \textbullet Implemented a sun tracking algorithm for satellite solar panels.\\
      \textbullet Awarded 500\$ AWS Credit Points Prize.\\ 
      \textbullet Navigate: \href{\github/Apollo}{\underline{Code}}
    }

\vspace{2mm}
\newpage

\Title{Awards \& Honors}

\entry
    {July 2024}
    {SciPy 2024 Conference}  
    {SciPy, NumFocus}
    {Tacoma, WA, USA}
    {
        \textbullet Selected to attend SciPy Conference 2024 with full financial aid.\\
        \textbullet Recognized for impact and dedication in advancing open source projects.\\
        \textbullet Engaged with leading experts in the field of scientific and high performance computing.\\
    }
% \newpage    

\entry
    {Summer 2022}
    {Top 100 entries \& Top 25 Articles}
    {Summer of Math Exposition (SoME\#2)}
    {3Blue1Brown \& Leios Labs}  
    {
        \textbullet Participated in a competition aimed at presenting in-depth content in mathematics, computer science, and physics through engaging mediums.\\
        \textbullet Secured a spot among the top 100 overall submissions.\\
        \textbullet Ranked in the top 25 for non-video submissions (e.g., articles and games).\\
    }

\vspace{4mm}


\Title{Open-Source Contributions}

\vspace{-2mm}
\begin{figure}[h]
\begin{tabular}{ l l}
- \bf{\href{https://github.com/networkx/networkx/pulls?q=is%3Apr+author%3Amohamedrezk122+}{\underline{NetworkX}}} & a network analysis library and graph theoretic algorithms in Python. \\
- \bf{\href{https://github.com/sympy/sympy/pulls?q=is%3Apr+author%3Amohamedrezk122+}{\underline{SymPy}}} & a computer algebra \& symbolic computation in Python.
\end{tabular}
\end{figure}

\vspace{3mm}




% \entry
%     {Summer 2021}
%     {NASA Space Apps Cairo}
%     {Virtual} 
%     {}
%     {
%       \textbullet Created a web interface to simulate distant rotating Jupiter asteroids.\\
%       \textbullet Implemented a real-time algorithm for computing light curves.\\
%       \textbullet Built a simplified 3D solar system scene using Blender's Python API to simulate the realistic conditions an asteroid encounters. \\
%       \textbullet Explore the project: \href{\github/Asteroid-Bent}{\underline{Code}}
%     }

\end{document}
