\documentclass[hidelinks]{report}
\usepackage{cv}

\begin{document}

\noindent
\begin{minipage}{.4\textwidth}
    \HUGE{\fontfamily{lmss} \selectfont \bfseries Mohamed \\ El Shorbagy}
    \vspace{-1mm}
\end{minipage}%
\hfill\hspace{.2\textwidth}
\begin{minipage}{.5\textwidth}
 \large
    2002114@eng.asu.edu.eg \\[.2\baselineskip]
    %% mohrizq895@gmail.com    \\ %
    $+20 \ 1222448102$ \\[.2\baselineskip]
    \url{\github} \\[.2\baselineskip]
    \url{\linkedin} \\ %
    \url{\blog} 
\end{minipage}


\vspace{2mm}

\large 
\Title{Education}

\entry
    {2020 - present} 
    {Ain Shams University, \normalfont Cairo, Egypt  }
    {Bachelor of Science in Computer Engineering }
    {\textbullet Senior year 1 (year 4/5) | Anticipated Graduation: Jul 2025 \\ 
     \textbullet CGPA: 3.67 / 4.0. \\  
     \textbullet \textbf{Relevant Courses:} Compiler Design, Discrete Math, Image Processing, Deep Learning,
     Operating Systems, Database Systems, Artificial Intelligence, Software Engineering, Distributed Systems, 
     Algorithms. 
    }  

\entry
    {Class 2020}
    {Dakahlia STEM School, \normalfont Gamasa, Egypt}
    {High School Certificate}
    {
    \textbullet CGPA: 4.0 / 4.0 (1$^{st}$ Rank)
    }

\vspace{2mm}

\Title{Experience}

\entry
    {Oct 2023 - Present }
    {ASMARINE (Autonomous Underwater Vehicles team), \normalfont ASU  }
    {AI Software Engineer}
    {
      \textbullet Implemented state-of-the-art algorithms in Computer Vision and SLAM. \\
      \textbullet Explored the feasibility of machine learning-based control in AUVs. \\
      \textbullet Optimized code for resource-constrained computers.
    }

\entry
    {Jun - Sep 2023}
    {Human Centered Mechatronics Lab, \normalfont ASU Virtual Hospitals }
    {Research Assistant}
    {
      Supervised by Dr. Mohamed Awad \\
      \textbullet Implemented a TCP communication tunnel to retrieve data from various sensors through XML commands. \\
      \textbullet Synchronized motion capture cameras with the metabolic energy consumption system. \\
      \textbullet Automated sensor calibration process.
    }

\entry
    {Aug - Oct 2022}
    {Center for Sound, Vibration \& Smart Structures \\
      \normalfont Dynamical Systems \& Digitalisation Cluster, Ain Shams University \vspace{1mm}
    }
    {Signal Processing \& Optimization Intern}
    {
      Supervised by Dr. Ahmed Hesham\\
      \textbullet Implemented an acoustic wave velocity algorithm within composites.\\
      \textbullet Analyzed raw sensor data to localize acoustic sources such as cracks. \\
      \textbullet Implemented optimization algorithms (particle swarm, simulated annealing, simplex method).\\
      \textbullet Navigate: 
      \href{\github/AE-software.git}{\underline{Code}} / %
      \href{\github/AE-software/blob/master/Final-Report/Report.pdf}{\underline{Report}}
    }

\vspace{2mm}

\Title{Travel Grants}

\entry
    {Mar 2023}
    {Cardiff University, School of Engineering, \normalfont Wales, UK}
    {Visiting Student}
    {
      Hosted by Dr. John McCrory \\
      Full funded program (UK-HE Climate Research Grant): \\
      \textbullet Cultural and educational exchange. \\
      \textbullet Conducted Fourier analysis and signal processing on composite structures. \\ 
    }

\vspace{2mm}
\newpage 
\Title{Articles}

\entry
    {Jul 2023}
    {The Generalization of Fifteen Puzzle as PQ Puzzle}  
    {} 
    {
      \textbullet Discussed the generalization and solvability of the fifteen puzzle.\\
      \textbullet Implemented a dedicated solver using weighted iterative deepening A*.\\
      \textbullet Navigate:  \href{\blog/fifteen-puzzle}{\underline{Article}} | 
      \href{\github/fifteen-puzzle-solver}{\underline{Solver Code}}
    } 


\entry
    {May 2022}
    {Diffusion Equation: A computational approach}  
    {} 
    {
        \textbullet Discussed diffusion equation as a mathematical, and computational
        model trying to generalize the results to higher dimensions \\
        \textbullet Navigate: \href{\blog/diffusion}{\underline{Article}}
    }

\vspace{2mm}

\Title{Personal Projects}


\entry
    {Nov 2023}
    {automata-cli}
    {}
    {
    \textbullet A command-line interface program to parse program-like automata specifications. \\ 
    \textbullet The stored automata structure can be minimized, converted to other forms, or 
    manipulated with user-specified algorithms. \\ 
    \textbullet Supports rendering automata to various formats for document embedding.\\
    \textbullet Navigate: \href{\github/automata-cli}{\underline{Code}}
    }

\entry
    {Feb 2023}
    {cv.py}
    {}     
    {
      \textbullet A CLI abstraction of the LaTeX-based CV building process using YAML to TeX conversion.\\
      \textbullet Compilation is done with a LaTeX cloud compiler or locally if applicable.
      \textbullet Navigate: \href{\github/CV.py}{\underline{Code}}
    }

\entry
    {Feb 2023}
    {Implementation of A* on Open Street Maps Data }
    {}     
    {
       \textbullet An implementation of the A* algorithm with KDTree and KNN algorithms to compute the shortest path on open street maps data based on the haversine heuristic and randomized median of medians algorithm to speed up KDTree formation.\\
       \textbullet This algorithm is developed on top of an extended Kalman filter that runs on a bare-metal embedded microcontroller with a GPS \& IMU. \\
       \textbullet Navigate: \href{\github/Astar-OSM}{\underline{Code}}
    }


\vspace{2mm}

\Title{Awards \& Honors}

\entry
    {Summer 2023}
    {NASA Space Apps Cairo, \normalfont The American University in Cairo }  
    {Most Innovative Solution with 25k Egyptian Pound Prize}
    {
      My team and I were fortunate to receive the "Most Innovative Solution" award along with a prize of 25,000 Egyptian pounds at the NASA Space Apps competition. This competition featured approximately 150 teams from various disciplines. Our project focused on the unique concept of data sonification, with a particular emphasis on enhancing the perception of space imagery.  
    }
    
\entry
    {Summer 2022}
    {Summer of Math Exposition (SoME\#2) | 3Blue1Brown \& Leios Labs}  
    {Top 100 entries \& Top 25 Articles}
    {
        \textbullet Secured one of the top 100 submissions overall.\\
        \textbullet Secured one of the top 25 non-video submissions (e.g. articles and games).\\
        \textbullet The competition focused on effectively presenting in-depth content in mathematics, computer science, and physics through engaging mediums.\\
        \textbullet Navigate: \href{\blog/diffusion}{\underline{Article}}
    }


\vspace{2mm}

\Title{Open-Source Contributions}
    
\begin{tabular}{ l l}
- \bf{\href{https://github.com/networkx/networkx/pulls?q=is%3Apr+author%3Amohamedrezk122+}{\underline{NetworkX}}} & a network analysis library and graph theoretic algorithms in Python. \\
- \bf{\href{https://github.com/sympy/sympy/pulls?q=is%3Apr+author%3Amohamedrezk122+}{\underline{SymPy}}} & a computer algebra \& symbolic computation in Python.
\end{tabular}

\newpage 

\vspace{2mm}

\Title{Hackathons \& Competitions}

\entry 
    {Summer 2023}
    {NASA Space Apps Cairo, \normalfont The American University in Cairo  }
    {}
    { \textbullet Developed a web application that transforms various types of space images, into audible sound and harmonious music. \\
      \textbullet Implemented a melody fitting algorithm: aligning classical music pieces with the input image.\\
      \textbullet Implemented a panoramic projection algorithm.\\
    }
    
\entry 
    {Summer 2022}
    {NASA Space Apps Cairo, \normalfont The American University in Cairo }
    {}
    { \textbullet Developed a web interface for ISS 3D virtual tracking in real-time. \\
      \textbullet Implemented orbital propagation algorithm for International Space Station.\\ 
      \textbullet Implemented a sun tracking algorithm for satellite solar panels.\\
      \textbullet Awarded 500\$ AWS Credit Points Prize.\\ 
      \textbullet Navigate: \href{\github/Apollo}{\underline{Code}}
    }


\entry
    {Summer 2021}
    {NASA Space Apps Cairo, \normalfont Virtual} 
    {}
    {
      \textbullet Created a web interface for simulating distant rotating Jupiter asteroids.\\
      \textbullet Implemented a real-time light curves computation algorithm \\
      \textbullet Constructed a simplified 3D solar system scene with Blender's Python API to simulate approximate conditions an asteroid encounters in reality. \\
      \textbullet Navigate: \href{\github/Asteroid-Bent}{\underline{Code}}
    }

\vspace{2mm}
    

\Title{Technical Experience}

\begin{tabular}{ l l l l l}
      \bf{- Programming-uages (high proficiency):} & Python, & Lua \\ 
      \bf{- Programming Languages (some proficiency):} & C/C++, & Java \\ 
      \bf{- Scientific Computing:} & Octave, & NumPy, & SciPy, & Matplotlib \\ 
      \bf{- System and Scripting:} & Linux, & Bash \\ 
      \bf{- Databases \& Data stores :} & MySQL, & TinyDB, & Pandas \\ 
       
\end{tabular}

\end{document}
