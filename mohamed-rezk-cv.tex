\documentclass[hidelinks]{report}
\usepackage{cv}
\usepackage{lmodern}
\pagestyle{empty}

\begin{document}


\noindent
\begin{minipage}{.4\textwidth}
    \HUGE{\fontfamily{lmss} \selectfont \bfseries Mohamed Rezk}
    \vspace{-1mm}
\end{minipage}%
\hfill\hspace{.2\textwidth}
\begin{minipage}{.5\textwidth}
 \large
    2002114@eng.asu.edu.eg \\[.2\baselineskip]
    %% mohrizq895@gmail.com    \\ %
    $+20 \ 1222448102$ \\[.2\baselineskip]
    \url{\github} \\[.2\baselineskip]
    \url{\linkedin} \\ %
    \url{\blog} 
\end{minipage}


\vspace{2mm}

\large 
\Title{Education}

\entry
    {2020 - present} 
    {Ain Shams University, \normalfont Cairo, Egypt  }
    {Bachelor of Science in Electrical \& Computer Engineering }
    {Senior year 1 (4/5)\hspace{2mm} \textbar \hspace{2mm} CGPA: 3.70 / 4.0 | Anticipated Graduation : Jul 2025}  

\entry
    {Class 2020}
    {Dakahlia STEM School, \normalfont Gamasa, Egypt}
    {High School Certificate}
    {CGPA: 4.0 / 4.0 (1$^{st}$ Rank)}

\vspace{2mm}

\Title{Experience}

\entry
    {Oct 2023 - Present }
    {ASMARINE (Autonomous Underwater Vehicles team), \normalfont ASU  }
    {Software Engineer}
    {
      Led by Eng. Samer Mansour \\
      \textbullet Implemented state-of-the-art algorithms in Computer Vision and SLAM. \\
      \textbullet Explored the feasibility of machine learning-based control in AUVs. \\
      \textbullet Optimized code for resource-constrained computers.
    }

\entry
    {Jun - Sep 2023}
    {Human Centered Mechatronics Lab, \normalfont ASU Virtual Hospitals }
    {Research Assistant}
    {
      Supervised by Dr. Mohamed Awad \\
      \textbullet Implemented a TCP communication tunnel to retrieve data from various sensors through XML commands. \\
      \textbullet Synchronized motion capture cameras with the metabolic energy consumption system. \\
      \textbullet Automated sensor calibration process.
    }


\entry
    {Mar 2023}
    {Cardiff University, School of Engineering, \normalfont Wales, UK}
    {Visiting Student}
    {
      Hosted by Dr. John McCrory \\
      Full ride program (UK-HE Climate Research Grant): \\
      \textbullet Cultural and educational exchange. \\
      \textbullet Conducted Fourier analysis and signal processing on composite structures. \\ 
      \textbullet Visited the Riverside community for sustainability and clean energy. 
    }

\entry
    {Aug - Oct 2022}
    {Center for Sound, Vibration \& Smart Structures \\
      \normalfont Dynamical Systems \& Digitalisation Cluster, Ain Shams University \vspace{1mm}
    }
    {Signal Processing \& Optimization Intern}
    {
      Supervised by Dr. Ahmed Hesham\\
      \textbullet Implemented an acoustic wave velocity algorithm within composites.\\
      \textbullet Analyzed raw sensor data to localize acoustic sources such as cracks. \\
      \textbullet Implemented optimization algorithms (particle swarm, simulated annealing, simplex method).
      \href{\github/AE-software.git}{\underline{Code}} / %
      \href{\github/AE-software/blob/master/Final-Report/Report.pdf}{\underline{Report}}
    }


\entry
    {Mar - Apr 2022}
    {CAMTECH CNC, \normalfont Maadi, Egypt}
    {Programmatic Video Editor}
    {
      Utilized FFmpeg for bulk conversion and employed DaVinci Resolve and Python to edit and create high-quality videos.
    }

\vspace{2mm}
\newpage
\Title{Articles}

\entry
    {Jul 2023}
    {The Fifteen Puzzle}  
    {} 
    {
      An article discussing the generalization and solvability of the fifteen
      puzzle and how to solve one if feasible using structured algorithmic techniques.
      I also implemented a dedicated solver to assist the article using weighted iterative
      deepening A* with some optimization under the hood. \href{\blog/fifteen-puzzle}{\underline{Article}} \textbar
      \href{\github/fifteen-puzzle-solver}{\underline{Code}}
    } 


\entry
    {May 2022}
    {Diffusion Equation: A computational approach}  
    {} 
    {
        An article discussing diffusion equation as a mathematical, and computational
        model. The article starts with introducing the math behind the physical model
        and tries to generalize the results to higher dimensions, taking into account
        the approximation nature of such a phenomenon.
        \href{\blog/diffusion}{\underline{Article}}
    }

\vspace{2mm}

\Title{Projects}

\entry
    {Feb 2023}
    {Implementation of A* on Open Street Maps Data }
    {}     
    {
       An Implementation of A* algorithm with KDTree and KNN algorithms to compute shortest path on a open street maps data based on haversine heuristic and randomized median of medians algorithm to speed up KDTree formation. This algorithm is developed on top of an extended Kalman filter that runs on a bare-metal embedded microcontroller with a GPS \& IMU. \href{\github/Astar-OSM}{\underline{Code}}
    }

\vspace{2mm}

\Title{Open-Source Contributions}
    
\entry
    {Nov 2023}
    {automata-cli}
    {}
    {
    A command-line interface program to render automata description in program-like form to a visual format, and also performing some algorithms on the automata structure like minimization and conversion to other forms.\href{\github/automata-cli}{\underline{Code}}
    }

\entry
    {Mar 2023}
    {bookmarks.sh}
    {}
    {
      A bash script for saving and retrieving bookmarks globally
      on UNIX-like OSs  based on demun software.\href{\github/bookmark.sh}{\underline{Code}}
    }
    
\entry
    {Feb 2023}
    {CV.py}
    {}     
    {
      A simple conversion of yaml file containing the CV information in a hierarchical
      order into a nice and elegant LaTeX file that is fed to an API for compilation. User does
      not need to know \LaTeX syntax nor have LaTeX compiler on his/her local machine.
      \href{\github/CV.py}{\underline{Code}}
    }

%% \entry
%%     {2021}
%%     {Maze solving algorithm} 
%%     {} 
    %% {An algorithm to compute the shortest path between two nodes in a maze with a graph
    %% theory approach- simulated and coded in python.\href{\github/Maze-solving-algorithm}{\underline{Code}}}

\vspace{2mm}


\Title{Awards \& Honors}

\entry
    {2023}
    {NASA Space Apps Cairo,}  
    {Most Innovative Solution \& 25k Egyptian Pound Prize}
    {
      My team and I were fortunate to receive the "Most Innovative Solution" award along with a prize of 25,000 Egyptian pounds at the NASA Space Apps competition. This competition featured approximately 150 teams from various disciplines. Our project focused on the unique concept of data sonification, with a particular emphasis on enhancing the perception of space imagery.  
    }
    
\entry
    {2022}
    {Summer of Math Exposition (SoME\#2) | 3Blue1Brown}  
    {Top 100 entries \& Top 25 Articles}
    {
        I secured a position within the top 100 participants at the Summer of Math Exposition (SoME\#2), an event organized by Grant Sanderson (3Blue1Brown) and Leios Labs. Additionally, my entry received recognition as one of the top 25 non-video submissions, covering diverse categories such as articles and games. The competition focused on effectively presenting in-depth content in mathematics, computer science, and physics through engaging mediums, including articles, games, and other creative formats. \href{\blog/diffusion}{\underline{Article}}
    }

\vspace{2mm}
\newpage
\Title{Hackathons \& Competitions}

\entry 
    {Summer 2023}
    {NASA Space Apps Cairo, \normalfont AUC }
    {}
    { Developed a web application that transforms various types of space images, into not only audible sound but also harmonious music. I did the following:\\
      \textbullet Implemented a melody fitting algorithm: aligning classical music pieces with the input image.\\
      \textbullet Implemented a  panoramic projection algorithm.\\
      \textbullet Awarded "Most Innovative Solution" with 25k L.E Prize.
    }
    
\entry 
    {Summer 2022}
    {NASA Space Apps Cairo, \normalfont AUC }
    {}
    { \textbullet Developed a web interface for ISS 3D virtual tracking in realtime \\
      \textbullet Implemented orbital propagator algorithm for International Space Station based on
      the laws of orbital mechanics and astronomical algorithms. \\ 
      \textbullet Implemented a sun tracking algorithm for stallite solar panels.\\
      \textbullet Awarded 500\$ AWS Credit Points Prize. \href{\github/Apollo}{\underline{Code}}
    }


\entry
    {Summer 2021}
    {NASA Space Apps Cairo, \normalfont Virtual} 
    {}
    {
      \textbullet Created a web interface for simulating distant rotating Jupiter asteroids.\\
      \textbullet Implemented a real-time light curves computation algorithm \\
      \textbullet Utilized Blender's Python API to construct a simplified 3D solar system scene. The objective was to simulate approximate conditions an asteroid encounter for input into an algorithm.
      \href{\github/Asteroid-Bent}{\underline{Code}}
    }

\entry
    {2018}
    {Egypt Robot Challenge} 
    {Ideas Gym}
    {- Silver Medal \& Global nominee to China} 

\vspace{2mm}
    
\Title{Activities}

\entry
    {2018 - 2020}
    {Dakahlia STEM School} 
    {Technical Support at Fab Lab Team} 
    {
     \textbullet Provided help for those who are not familiar with Fab Lab machines\\
     \textbullet Gave a detailed background on Fab Lab policies and tools for newcomers and visitors.
    }
    
\entry
    {2019 - 2020}
    {Dakahlia STEM School}  
    {Mentor at DK Counseling Club} 
    {
     \textbullet Provided students with adequate knowledge about efficient ways of learning and creating.\\
     \textbullet Gave sessions and presentation about higher education options for school students.
    }

\entry
    {2019 - 2020}
    {Dakahlia STEM School}
    {Mentor at DK Robotics Club} 
    {\textbullet Helped students with their school projects and prototypes.}

\vspace{2mm}
    

\Title{Programming Skills}

\col
      {Python}          {\LaTeX}        {Octave} %
      {C/C++}           {Java}          {Programmatic animation}%
      {Bash scripting}  {Linux/UNIX}    {Lua}%

\vspace{2mm}
\Title{Languages}

\col 
    {Arabic}
    {English}
    {German (Fit in Deutsch A1)}
    {} {} {} {} {} {}
\end{document}
