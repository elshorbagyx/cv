\documentclass[]{report}
\usepackage{cv}

\begin{document}

\begin{center}
    \noindent \Huge{\bfseries Mohamed El Shorbagy}\\[.4em]
    \large
        \textbf{Email:} mohrizq895@gmail.com 
    |   \textbf{Tel:} $+201222448102$
    |   \link{GitHub}{\github}
    |   \link{LinkedIn}{\linkedin}
    |   \link{Website}{\site}   
\end{center}

\vspace{2mm}

\large
\Title{Education}

\entry
    {2020 - 2025} 
    {Bachelor of Science in Computer and Systems Engineering }
    {Ain Shams University}
    {Cairo, Egypt}
    {\tb \textbf{GPA}: 3.64 / 4.00 | \textbf{Class rank}: 4/160 \\  
     \tb \textbf{Relevant Courses:} Compiler Design, Operating Systems, Embedded Systems, Network Security, Discrete Math, Image Processing, Machine Learning, Deep Learning, Database Systems, Artificial Intelligence, Software Engineering, Distributed Systems, Data Structures, Design of Algorithms.\\
     \tb \textbf{Thesis:} \emph{\textbf{Automated Dental Crown Generation Pipeline}}, sponsored by \linkk{Atomica.ai}{https://atomica.ai/}\\
     % Contributions:\\
     \tb Developed \emph{mcg}: a minimal geometry processing library in \textbf{C++} with a focus on speed. \emph{mcg} is written to support graphics operations such as closest point computations and intersection tests, geometric operations such as mesh surface deformation (Laplacian deformer), halfedge-mesh operations such as enclosed faces inside loops, shortest path computations on the surface of mesh such as A*, and to provide spatial data structures such as BVHTree and KDTree.\\
     \tb Developed virtual dental arch alignment algorithm (using \emph{mcg}).\\
     \tb Developed automated dental crown positioning algorithm (using \emph{mcg}).\\
     \tb Developed semi-automatic dental restoration pipeline (using \emph{mcg}).\\
     \tb Developed \emph{fastmesh}: a lazy mesh file parser and optimizer in \textbf{Rust} with Python bindings (using PyO3) for accelerated mesh operations to serve 3D deep learning based tasks.
    }  

\vspace{2mm}

\Title{Technical Skills}

\begin{tabular}{ l l l l l l l}
      \bf{- Programming-Languages:} & Python, & C++, & Lua, & Rust, & Zig\\ 
      \bf{- Graphics:} & OpenGL,&  SDL, & ASSIMP \\ 
      \bf{- Tools \& Platforms:} & Linux, & Bash, & CMake, & git, & gdb, & perf \\ 
      % \bf{- Scientific Computing:} & Octave, & NumPy, & SciPy, & Matplotlib \\ 
      % \bf{- Databases \& Data stores :} & MySQL, & TinyDB, & Pandas \\ 
       
\end{tabular}

\vspace{4mm}

\Title{Experience}

\entry
    {Feb 2025 - Present}
    {Computational Geometry Software Engineer (C++)}
    {Atomica AI (Dental CAD software suite), Remote}
    {Atlanta, GA, United States}
    {
        \tb Optimized BVHTree reducing memory consumption by 50\% and build time.\\
        \tb Worked on adaptive remeshing to reduce triangle count in planar areas.\\
        \tb Implemented balanced KDTree data structure to support pointcloud operations.\\
        \tb Worked on repair algorithms such as non-manifold vertex handling, collapsing, and hole stitching.\\
        \tb Worked on dental crown generation tools with Laplacian surface deformer.\\
    }

\entry
    {Oct 2023 - Feb 2024}
    {Software Engineering Intern}
    {ASMARINE (Autonomous Underwater Vehicles team)}
    {ASU, Cairo, Egypt}
    {
      \tb Implemented state-of-the-art algorithms for SLAM and computational geometry.\\
      \tb Optimized code for resource-constrained computers.
    }

\entry
    {Jun - Sep 2023}
    {Undergraduate Research Assistant}
    {Human-Centered Mechatronics Lab}
    {ASU Virtual Hospitals, ASU}
    {
      \tb Implemented a TCP communication tunnel to retrieve sensor data via XML communication. \\
      \tb Synchronized motion capture cameras with metabolic energy measurement systems. \\
    }

\entry
    {Aug - Oct 2022}
    {Optimization \& Signal Processing Intern}
    {Dynamic Systems \& Digitalization cluster - Cardiff University}
    {ASU, Cairo}
    {
      \tb Utilized the Akaike Information Criterion estimator for precise determination of signal onset time.\\
      \tb Implemented TDOA algorithm with particle swarm optimization to localize acoustic sources.
    }

\newpage
\Title{Personal Projects}

\entry
    {Jul 2024 - present}
    {zain: Toy 64-bit Lua VM interpreter implementation in Zig}
    {}{}
    {
        \tb Implemented high-performance Lua lexer achieving $\approx$ 120 MB/s throughput.\\
        \tb Implemented a recursive descent parser with precedence climbing algorithm.\\
        \tb Developed a Lua 5.3 bytecode decompiler.
    }

% \entry
%     {Sep 2024 - present}
%     {mesher: 3D Triangular mesh viewer in C++, \link{Code}{\github/mesher}}
%     {}{}
%     {
%         \tb Implemented a mesh viewer with OpenGL API, supporting lighting, and camera movement.\\
%         \tb Implemented triangle selection algorithm with BVH and Möller–Trumbore algorithm.
%     }

\entry
    {Aug 2024}
    {mark: CLI-based bookmark manager, \link{Code}{\github/mark}}
    {} {}
    {
        \tb Implemented client-server architecture with synchronous sockets for Rofi integration.\\
        \tb Created a wrapper around TinyDB with orjson for efficient bookmark storage and querying.\\ 
        \tb Added a parser for the Netscape bookmark file format and various export options.
    }

\entry
    {Nov 2023}
    {automata-cli: Automata Renderer and Minimizer, \link{Code}{\github/automata-cli}}
    {} {}
    {
        \tb Built a CLI tool to parse and manipulate program-like automata specifications. \\ 
        \tb Enabled minimization, format conversion, and custom algorithm manipulation. \\ 
        \tb Supported rendering automata into various formats for document embedding.
    }

\entry
    {Feb 2023}
    {cv.py: YAML to LaTeX Adapter, \link{Code}{\github/CV.py}}
    {} {}  
    {
        \tb Created a CLI tool to easily convert YAML files into LaTeX-based CVs.\\
        \tb Enabled users to focus on content creation while the tool manages the formatting process.\\
        \tb Supported CV compilation through either a cloud-based LaTeX compiler or local compilation.
    }

% \entry
%     {Feb 2023}
%     {Implementation of A* on Open Street Maps Data, \link{Code}{\github/Astar-OSM}}
%     {} {}    
%     {
%        \tb Implemented A* algorithm with KDTree and KNN to compute the shortest path on OSM data.\\
%        \tb Utilized the randomized median of medians algorithm to accelerate KDTree construction. \\
%        \tb Developed the algorithm on top of an extended Kalman filter and embedded microcomputer.
%     }

\Title{Open-Source Contributions}

\noindent I contributed to the following software/packages; follow the links for more details.\\
\begin{tabular}{lll}

- \linkk{Blender}{https://projects.blender.org/blender/blender/pulls?type=created_by\&state=closed\&milestone=0\&project=0\&assignee=0\&poster=elshorbagyx}
 & (1 Merged) 
 & a 3D creation suite and graphics software in C++. \\
 
 - \linkk{PMP-Library}{https://github.com/pmp-library/pmp-library/pulls?q=\%20is\%3Apr\%20author\%3Aelshorbagyx} 
 &(1 Merged)
 & a polygon mesh geometry processing library in C++.\\
- \linkk{NetworkX}{https://github.com/networkx/networkx/pulls?q=is\%3Apr+author\%3Aelshorbagyx+}
 &(6 Merged \& 2 Closed)
 & a network analysis and graph theoretic algorithms in Python. \\
- \linkk{SymPy}{https://github.com/sympy/sympy/pulls?q=is\%3Apr+author\%3Aelshorbagyx+}
 &(2 Merged \& 3 Open)
 & a computer algebra \& symbolic computation in Python.
\end{tabular}

\vspace{4mm}

\Title{Hackathons \& Competitions}

\entry
    {Summer 2023}
    {NASA Space Apps Cairo}
    {The American University in Cairo }{}
    {
        \tb Developed a project focused on data sonification, enhancing the perception of space imagery.\\
        \tb Implemented a melody fitting algorithm for aligning classical music pieces with the input image.\\
        \tb Received the \textbf{"Most Innovative Solution"} award and \textbf{25,000 EGP} prize.
    }
    
\entry 
    {Summer 2022}
    {NASA Space Apps Cairo}
    {The American University in Cairo }{}
    { 
        \tb Developed a web interface for ISS 3D virtual tracking in real-time. \\
        \tb Implemented orbital propagation algorithm and a sun tracking algorithm for ISS solar panels.\\
        \tb Awarded \$500 AWS Credit.
    }

\Title{Awards \& Honors}

\entry
    {July 2024}
    {SciPy 2024 Conference on High Performance Computing}  
    {SciPy, NumFocus}
    {Tacoma, WA, USA}
    {
        \tb Selected as one of \textbf{16 scholarship} recipients (out of \textbf{638 attendees}) to attend SciPy 2024 Conference with full financial aid.\\
        % \tb Engaged with leading experts in scientific and high-performance computing at the premier conference for the SciPy ecosystem.\\
        \tb SciPy libraries (e.g., NumPy, NetworkX, SymPy, etc.) power scientific computing across industries including aerospace (NASA, SpaceX), pharmaceuticals (drug discovery), finance (quantitative analysis), and technology (Google, Netflix data processing).
    }

\entry
    {Summer 2022}
    {Top 100 entries \& Top 25 Articles}
    {Summer of Math Exposition (SoME\#2)}
    {3Blue1Brown \& Leios Labs}  
    {
        \tb Participated in a global competition for creating in-depth math, CS, and physics content.\\
        \tb Secured a spot among the \textbf{top 100} overall out of $\approx$ \textbf{550 submissions}.\\
        \tb Ranked in the \textbf{top 25} for non-video submissions (e.g., articles and games) out of $\approx$ \textbf{125 entries}.
    }
% \Title{Articles}

% \entry
%     {Jul 2023}
%     {The Generalization of Fifteen Puzzle, 
%     \link{Article}{\site/fifteen-puzzle}\link{Code}{\github/fifteen-puzzle-solver}}  
%     {} {}
%     {
%       \tb Discussed the generalization and solvability of the fifteen puzzle with group theory.\\
%       \tb Implemented a PQ solver using weighted iterative deepening A*.
%     } 

% \entry
%     {May 2022}
%     {Diffusion Equation: A computational approach, \link{Article}{\site/diffusion}}  
%     {} 
%     {}
%     {
%         \tb Analyzed the diffusion equation as both a mathematical and computational model.\\
%         \tb Worked on generalizing results to higher dimensions.    
%     }

\end{document}
